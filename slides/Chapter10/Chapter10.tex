%%%%%%%%%%%%%%%%%%%%%%%%%%%%%%%%%%%%%%%%%
% Beamer Presentation
% LaTeX Template
% Version 1.0 (10/11/12)
%
% This template has been downloaded from:
% http://www.LaTeXTemplates.com
%
% License:
% CC BY-NC-SA 3.0 (http://creativecommons.org/licenses/by-nc-sa/3.0/)
%
%%%%%%%%%%%%%%%%%%%%%%%%%%%%%%%%%%%%%%%%%

%----------------------------------------------------------------------------------------
%	PACKAGES AND THEMES
%----------------------------------------------------------------------------------------

\documentclass[aspectratio=43,UTF8,10pt,t]{ctexbeamer}

\mode<presentation> {
	\usetheme{Madrid}
	\setbeamertemplate{footline}{} % To remove the footer line in all slides
	\setbeamertemplate{navigation symbols}{} % To remove the navigation symbols from the bottom of all slides
}

% User Defined Block %%%%%%%%%%%%%%%%%%%%%%%%%%%%%%%%%%%%%%%%%%%%%%%%%%%%%%%%
\usepackage{multirow}

\usepackage{setspace}
\definecolor{hanblue}{rgb}{0.27, 0.42, 0.81}
\definecolor{indiagreen}{rgb}{0.07, 0.53, 0.03}
\definecolor{indianred}{rgb}{0.8, 0.36, 0.36}
\definecolor{indianyellow}{rgb}{0.89, 0.66, 0.34}
\definecolor{babypink}{rgb}{0.96, 0.76, 0.76}
\definecolor{ao(english)}{rgb}{0.0, 0.5, 0.0}
\setbeamerfont{block title}{size=\small}
\setbeamerfont{block body}{size=\footnotesize}
\newenvironment<>{blueblock}[1]{%
	\setbeamercolor{block title}{fg=white,bg=hanblue}%
	\begin{block}#2{#1}}{\end{block}}
\newenvironment<>{greenblock}[1]{%
	\setstretch{1.3}\setbeamercolor{block title}{fg=white,bg=indiagreen}%
	\begin{block}#2{#1}}{\end{block}}
\newenvironment<>{redblock}[1]{%
	\setstretch{1.3}\setbeamercolor{block title}{fg=white,bg=indianred}%
	\begin{block}#2{#1}}{\end{block}}
\newenvironment<>{yellowblock}[1]{%
	\setstretch{1.3}\setbeamercolor{block title}{fg=white,bg=indianyellow}%
	\begin{block}#2{#1}}{\end{block}}

%----------------------------------------------------------------------------------------
%	PACKAGES
%----------------------------------------------------------------------------------------
\usepackage{graphicx} % Allows including images
%\usepackage{tikz}
%\usetikzlibrary{shapes.geometric, arrows}
\usepackage{listings}
\lstset{language=C++,
	columns=flexible,
	basicstyle=\footnotesize\ttfamily,                                      % 设定代码字体、大小
	%numbers=left,xleftmargin=2em,framexleftmargin=2em,                   % 在左侧显示行号
	%numberstyle=\color{darkgray},                                        % 设定行号格式
	keywordstyle=\color{blue},                                            % 设定关键字格式
	commentstyle=\color{ao(english)},                                     % 设置代码注释的格式
	stringstyle=\color{brown},                                            % 设置字符串格式
	%showstringspaces=false,                                              % 控制是否显示空格
	%frame=lines,                                                         % 控制外框
	breaklines,                                                           % 控制是否折行
	postbreak=\space,                                                     % 控制折行后显示的标识字符
	breakindent=5pt,                                                      % 控制折行后缩进数量
	emph={size\_t,array,deque,list,map,queue,set,stack,vector,string,pair,tuple}, % 非内置类型
	emphstyle={\color{teal}},
	escapeinside={(*@}{@*)},
}

%----------------------------------------------------------------------------------------
%	TITLE PAGE
%----------------------------------------------------------------------------------------

\title[\textit{C++程序设计:第十章}]{第十章~简单输入输出} % The short title appears at the bottom of every slide, the full title is only on the title page

%\author[李长河]{李长河} % Your name
%\institute[CUG] % Your institution as it will appear on the bottom of every slide, may be shorthand to save space
%{
%中国地质大学(武汉)\\ % Your institution for the title page
%\medskip
%\textit{lichanghe@cug.edu.cn} % Your email address
%}
\date{} % Date, can be changed to a custom date

\begin{document}
	
	%----------------------------------------------------------------------------------------
	%	TIKZ FLOWCHART
	%----------------------------------------------------------------------------------------
	%\tikzstyle{startstop} = [rectangle, rounded corners, minimum width=2cm, minimum height=0.5cm, text centered, draw=black, fill=red!30, font=\tiny]
	%\tikzstyle{io} = [trapezium, trapezium left angle=70, trapezium right angle=110, minimum width=0cm, minimum height=0cm, text centered, draw=black, fill=blue!30, font=\tiny]
	%\tikzstyle{process} = [rectangle, minimum width=2.5cm, minimum height=1.5cm, text centered, draw=black, fill=orange!30, font=\tiny, text width=2cm]
	%\tikzstyle{decision} = [diamond, minimum width=2.5cm, minimum height=2cm, text centered, draw=black, fill=green!30, font=\tiny, text width=1.8cm, aspect=1.1]
	
	\begin{frame}
		\titlepage % Print the title page as the first slide
	\end{frame}
	
	\begin{frame}{目录}
		\tableofcontents
	\end{frame}
	
	%----------------------------------------------------------------------------------------
	%	PRESENTATION SLIDES
	%----------------------------------------------------------------------------------------
	
	%--------------------
	
	\begin{frame}[fragile]{~} % Table of contents slide, comment this block out to remove it
		
		\begin{block}{学习目标}
			\begin{enumerate}
				\item 了解常用~IO~类的继承关系和理解~IO~流基本工作流程;
				\item 掌握常见的输入输出格式控制;
				\item 掌握文件流和~string~流的使用方法。
			\end{enumerate}
		\end{block}
		
		% ------功能模块说明,请注释掉------- 
		\begin{columns}[t]
			\column{0.15\textwidth}
			\begin{block}{概念}
			\end{block}
			\column{0.15\textwidth}
			\begin{blueblock}{代码}
			\end{blueblock}
			\column{0.15\textwidth}
			\begin{yellowblock}{说明}
			\end{yellowblock}
			\column{0.15\textwidth}
			\begin{greenblock}{问题/答案}
			\end{greenblock}
			\column{0.15\textwidth}
			\begin{redblock}{注意}
			\end{redblock}
		\end{columns}
		% ------功能模块说明,请注释掉------- 
		
	\end{frame}
	
	%--------------------
	
	%#####################################
	\section{基本知识}
	%#####################################
	
	\begin{frame}[fragile]{10.1~基本知识}
		
		\begin{block}{C++的IO操作}
			C++~语言不能直接处理~IO~操作,依靠不同的~IO~类来实现从设备中读取数据和向设备写入数据。\\
			\vspace{4ex}
			例如之前用到的:~cin、~cout、~\verb;>>;~。
		\end{block}

		

		
	\end{frame}
	
	%-------------------------------------
	\subsection{~IO~类对象}
	%-------------------------------------
	
	\begin{frame}[fragile]{10.1.1~IO~类对象}
		\begin{block}{流(stream)}
		数据从数据源到目的端的流动过程称之为\alert{流}。
		\end{block}
		\vspace{4ex}
		\begin{block}{常用的~IO~类}
		常用的~IO~类有~itream~和~ostream~,同时包括文件流类型和~string~流类型。\\
		\vspace{2ex}
		ios~是抽象基类,由它派生出~istream~类和~ostream~类,两个类名中第1个字母~i~和~o~分别代表输入(input)和输出(output)。ifstream~和~ofstream~类用于文件输入输出,类名中第2个字母~f~代表文件(file)。\\
		\vspace{2ex}
		类型~ifstream~和~istringstream~继承自~istream,类型~ofstream~和~ostringstream~继承自~ostream。
		\end{block}
	

	\end{frame}

	\begin{frame}[fragile]{10.1.1~IO~类对象}
	常用IO类库简介:
	\begin{table}[t]
		\begin{center}\ttfamily
			\color{hanblue}\color{hanblue}\label{tab10-1}\color{hanblue}
			\begin{tabular}{llp{0.6\textwidth}}\hline
				头文件                    & 类名          & 功能                                 \\\hline
				iostream                  & ios           & 抽象基类                             \\\hline
				\multirow{3}{*}{iostream} & istream       & 通用输入流和其他输入流的基类         \\
				& ostream       & 通用输出流和其他输出流的基类         \\
				& iostream      & 通用输入输出流和其他输入输出流的基类 \\\hline
				\multirow{3}{*}{fstream} & ifstream      & 输入文件流类                         \\
				& ofstream      & 输出文件流类                         \\
				&fstream& 输入输出文件流类                     \\\hline
				\multirow{3}{*}{sstream}  & istringstream & 输入字符串流类                       \\
				& ostringstream & 输出字符串流类                       \\
				& stringstream  & 输入输出字符串流类                   \\\hline
			\end{tabular}
		\end{center}
	\end{table}
	
	\end{frame}
	
	\begin{frame}[fragile]{10.1.1~IO~类对象}
		\begin{block}{常用对象~cin~,~cout~:}
		cin~是~istream~类的对象,它从标准输入设备(键盘)获取数据,通过输入运算符~\verb;>>;~从流中提取数据,提取的时候会根据对象的类型从输入流中提取相应长度的字节,cin\verb;>>;~从流中提取数据时通常跳过输入流中的空格、制表符、换行符等空白字符。\\
		\vspace{2ex}
		cout~是~ostream~类对象,它向控制台窗口输出数据。
		\end{block}
		\vspace{4ex}
		\begin{redblock}{和普通对象的区别:}
		和普通对象不同,IO~对象不支持赋值操作。
		\end{redblock}
	
		\begin{blueblock}<2->{示例:}
			\begin{lstlisting}[moreemph={Array}]
            ifstream in1, in2;//定义两个文件输入流对象
            in1 = in2; //错误:不能对流对象赋值
            //同样,IO对象也不支持复制操作:
            ostream print(ostream);//错误:不能按值方式返回或传递ostream对象
        \end{lstlisting}
		\end{blueblock}
	\end{frame}
	%-------------------------------------
	\subsection{条件状态}
	\begin{frame}[fragile]{10.1.2状态条件}
		\begin{blueblock}{请看如下情况:}
			\begin{lstlisting}[moreemph={Array,T,Less,F}]
            double x;
            cin>>x;
            \end{lstlisting}
		\end{blueblock}
		
		\begin{greenblock}<2->{}
			当输入为char类型的字符,会怎么样呢?
		\end{greenblock}
		\begin{greenblock}<3->{}
			代码中的~cin\verb;>>;~期待读取一个~double,但却提供了字符数据,cin~会进入错误状态。一旦~cin~进入到错误状态,它就变成无效的,无法再执行后续的输入。因此,在使用~cin~时,要确保它的状态是有效的。
		\end{greenblock}

	\end{frame}
	\begin{frame}[fragile]{10.1.2状态条件}
		
		\begin{greenblock}<1->{}
			那么又该如何来判断状态的有效性呢?
		\end{greenblock}
		\begin{blueblock}{}<2->
			\begin{lstlisting}[moreemph={Array,T,Less,F}]
            while(cin>>x)//遇到错误状态循环将退出;
            if(!cin)cin.clear();//clear()函数执行后,cin变为有效状态;
            \end{lstlisting}

		\end{blueblock}
		
	\end{frame}
	\subsection{刷新缓冲区}
	\begin{frame}[fragile]{10.1.3刷新缓冲区}
		\begin{blueblock}{缓冲区刷新的原因:}
		导致缓冲区刷新有很多原因,比如缓冲区满、程序正常结束、遇到~endl~等。缓冲区刷新完成后,原来的数据被清空。
		\end{blueblock}
		\begin{blueblock}{示例:}
		\begin{lstlisting}[moreemph={Array,T,Less,F}]
            cout<<"endl"<<endl;//输出endl和一个换行,然后刷新缓冲区
            cout<<"flush"<<flush;//输出flush(无额外字符),然后刷新缓冲区
            cout<<"ends"<<ends;//输出ends和一个空字符,然后刷新缓冲区
        \end{lstlisting}
		\end{blueblock}
	\end{frame}
	
	\section{标准输入输出}
	
	\begin{frame}[fragile]{10.2~标准输入输出}
		输入输出的控制是C++程序当中最常用的一些操作,字符数据的格式化控制尤为重要。
	\end{frame}
	\subsection{字符数据的输入}
	\begin{frame}[fragile]{10.2.1~字符数据的输入}
		\begin{blueblock}{使用>>运算符:}
			数据的输入一般以空白字符结束(包括空格符、制表符和回车符等),而这些空白字符会被系统过滤掉。
		\end{blueblock}
		\vspace{4ex}
		\begin{blueblock}{使用cin.get()函数:}
			cin.get()~函数可以获取空白字符。其功能是从输入流中获取一个字符,并将其返回。
		\end{blueblock}
	
		\begin{blueblock}{示例:}
            \begin{lstlisting}[moreemph={Array,T,Less,F}]
            for(char c;(c=cin.get())!='\n';)
               cout<<c;
            cout<<endl;            
            \end{lstlisting}
		\end{blueblock}

	\end{frame}
	
	\begin{frame}[fragile]{10.2.1~字符数据的输入}
		\begin{blueblock}{使用cin.getline()函数:}
			getline~函数以回车符作为输入结束的标志符,把从输入流~cin~中提取的字符序列(不包括回车符)放到~string~类对象~s~中,并返回~cin~的引用。 
		\end{blueblock}
		\begin{blueblock}{示例:}
			\begin{lstlisting}[moreemph={Array,T,Less,F}]
            string s;
            getline(cin,s);          
            \end{lstlisting}
		\end{blueblock}

	\end{frame}
	\subsection{格式化控制}
	\begin{frame}[fragile]{10.2.2~格式化控制}
		\begin{blueblock}{整形值的进制}
			默认格式按照十进制输入输出,其他进制可以用进制说明符进行转换。
		\end{blueblock}
		\begin{blueblock}{示例:}
			\begin{lstlisting}[moreemph={Array,T,Less,F}]
            cout<<showbase<<uppercase;//显示进制信息,十六进制数以大写形式输出
            cout<<"default:"<<26<<endl;
            cout<<"octal:"<<oct<<26<<endl;
            cout<<"decimal:"<<dec<<26<<endl;
            cout<<"hex:"<<hex<<26<<endl;
            cout<<noshowbase<<nouppercase<<dec;
            int i,j;
            cin>>oct>>i;//输入格式为八进制;
            cin>>hex>>j;//输入格式为十六进制;       
            \end{lstlisting}
		\end{blueblock}
	\begin{redblock}{注意:}<2->
		上面最后一条语句执行完之后,后续输入的数据为十六进制,用户可以利用~dec~将进制形式恢复为默认的十进制。
	\end{redblock}
	\end{frame}
	\begin{frame}[fragile]{10.2.2~格式化控制}
		\begin{blueblock}{控制打印精度:}
			\begin{lstlisting}[moreemph={Array,T,Less,F}]
            double x=1.2152;
            cout.precision(3);//使用precision成员函数指定打印精度;
            cout<<"precision:"<<cout.precision()<<",x="<<x<<endl;
            cout<<setprecision(4);//使用setprecision函数指定打印精度;
            cout<<"precision:"<<cout.precision()<<",x="<<x<<endl;
            cout<<"scientific:"<<scientific<<10*exp(1.0)<<endl;//用科学计数法控制输出格式;
            cout<<"fixed decimal:"<<fixed<<10*exp(1.0)<<endl;//定点十进制默认格式;
            cout<<"default float:"<<defaultfloat<<10*exp(1.0)<<endl;
            
            \end{lstlisting}
		\end{blueblock}
	\vspace{4ex}
	\begin{redblock}{注意:}<2->
		在执行~scientific~或者~fixed~操纵符后,精度控制的是小数点后面的数值位数,而不是默认的数值总位数。defaultfloat~为~C++11~新特性,它将流恢复到默认状态。
	\end{redblock}
	\end{frame}
	\begin{frame}[fragile]{10.2.2~格式化控制}
		\begin{blueblock}{利用~setw~指定占用宽度}
			\begin{lstlisting}[moreemph={Array,T,Less,F}]
            int i=-10;
            double x=1.2152;
            cout<<"i:"<setw(10)<<i<<endl;
            cout<<"x:"<<setw(10)<<x<<endl;
            输出结果:
            i:          -10
            x:       1.2152
            \end{lstlisting}
		\end{blueblock}
	\end{frame}
\section{文件输入输出与~string~流}
	\begin{frame}[fragile]{10.3~文件输入输出与~string~流}
		\begin{block}{文件流:}
			和磁盘进行数据交换需要文件流,其中文件流包括:~iftream~,~ofstream~,~fstream~,它们分别可以从指定文件读取数据,向指定文件写入数据和读写文件。
		\end{block}
	\end{frame}

\subsection{使用文件流对象}
	\begin{frame}[fragile]{10.3.1~使用文件流对象}
		\begin{block}{文件流对象的创建和关联:}
			\begin{lstlisting}[moreemph={Array,T,Less,F}]
            ifstream in(ifname);//创建输入文件流对象,提供文件名;
            ofstream out;//创建输出文件流对象,没有提供文件名;
            \end{lstlisting}
		\end{block}
		\begin{block}<2->{文件流的打开与关闭:}
			\begin{lstlisting}[moreemph={Array,T,Less,F}]
                out.open(name);//调用open 函数,使之与一个文件关联;
                if(out);//用于检测open操作是否成功;
                out.close();//调用close函数关闭文件;
            \end{lstlisting}
		\end{block}
	\end{frame}
\subsection{文件模式}
	\begin{frame}[fragile]{10.3.2~文件模式}
		每个文件都有一些\alert{文件模式},用来指定如何使用文件。
		\vspace{4ex}
		\begin{blueblock}{常用的文件模式}
			\begin{itemize}
				\item ios::in  以读方式打开文件;
				\item ios::out	以写方式打开文件(默认方式)。如果已有此文件,则将其原有内容全部擦除,如文件不存在,则建立新文件;
				\item ios::app	以写方式打开文件,写入的数据追加到文件末尾;
				\item ios::ate	打开一个已有的文件,并定位到文件末尾;
				\item ios::binary	以二进制方式打开一个文件,如不指定此方式则默认为~ASCII~方式。
			\end{itemize}
		\end{blueblock}
		\begin{yellowblock}<2->{说明:}
			每一个文件流类型都设置了一个默认的文件模式,如果没有指定具体的文件模式,则以默认模式打开。ios::in~是~ifstream~流的默认模式,ios::out~是~ofstream~流的默认模式。fstream~的默认模式为~ios::in~和~ios::out。
		\end{yellowblock}
	\end{frame}
	\begin{frame}[fragile]{10.3.2~文件模式}
		文件读取示例:将百鸡问题中结果保存,然后读出计算结果并且打印输出。
		\begin{blueblock}

		\begin{lstlisting}[moreemph={Array,T,Less,F}]
        #include<iostream>
        #include<string>
        #include<iomanip>//使用setw函数
        #include<fstream>//文件输入输出
        using namespace std;
        int main() {
        int max_rst = 100 / 5, max_hen = 100 / 3;
        ofstream out("result.txt");//在当前目录创建文件;
        if (out) { //判断文件是否成功打开;
                 out <<setw(10)<<"公鸡"<<setw(10)<<"母鸡"<<setw(10)<< "小鸡";
                 for (int i = 0; i < max_rst; ++i) {
                      for (int j = 0; j < max_hen; ++j) {
                           int k = 100 - i - j;
                           if (k % 3) continue;
                           if (5 * i + 3 * j + k / 3 == 100)//向文件写入数据;
                               out<<'\n'<<setw(10)<<i<<setw(10)<<j<<setw(10)<<k;
                      }
                 }
                out.close();//关闭文件;
        }
        \end{lstlisting}

		\end{blueblock}
	\end{frame}
	\begin{frame}[fragile]{10.3.2~文件模式}
	文件读取示例:将百鸡问题中结果保存,然后读出计算结果并且打印输出。(\alert{续})
	\begin{blueblock}

	\begin{lstlisting}[moreemph={Array,T,Less,F}]

        ifstream in("result.txt");//打开当前目录下的文件;
        if (in) {//判断文件是否成功打开;
            string head;
            getline(in, head);
            cout << head << endl;
            int r[3];
            while (!in.eof()) {//成员函数eof用来判读文件流是否结束;             
                 in>>r[0]>>r[1]>>r[2];//从文件读取数据;
                 cout<<setw(10)<<r[0]<<setw(10)<<r[1]<<setw(10)<<r[2]<<endl;
           }
           in.close();//关闭文件;
        }
        return 0;
        }
     \end{lstlisting}
	\end{blueblock}
	\begin{yellowblock}{说明:}<2->
		在打开文件时,可以指定文件的具体路径,例如~“d:/result.txt”;如缺省路径,则默认为当前目录下的文件
	\end{yellowblock}
\end{frame}
\subsection{~string~流}
	\begin{frame}[fragile]{~string~流}
		string~流可以向~string~类对象写入数据,也可以从~string~类对象读取数据。string~流定义在~sstream~头文件中,它包含三个类型:istringstream、~ostringstream~和~stringstream。
		\vspace{4ex}
		\begin{blueblock}{string~流}
			\begin{itemize}
				\item istringstream~~~从~string~对象读取数据;
				\vspace{2ex}
				\item ostringstream~~~向~string~ 对象写入数据;
				\vspace{2ex}
				\item stringstream~~~ 既可以从~string~对象读取数据也可以向~string~对象写入数据。
			\end{itemize}
		\end{blueblock}
	\end{frame}

	\begin{frame}[fragile]{~istringstream~流}
		当从设备读取一行文本时,往往需要对整行文本中的单个单词进行处理,这时可以使用~istringstream~流对象。比如,需要获取一行文本中的所有单词,并把它们存放到一个~vector~里面。
		\vspace{4ex}
		\begin{blueblock}{使用示例:}<2->
				\begin{lstlisting}[moreemph={Array,T,Less,F}]
                vector<string>wds;//保存读取的单词;
                string line,word;
                while(getlien(cin,line)){
                    istringstream iss(line);//创建输入的string流对象,保存line的副本;
                    while(iss>>word)
                         wds.push_back(word);//将读取到的单词尾插;
                }
                \end{lstlisting}
		\end{blueblock}
	\end{frame}
\begin{frame}[fragile]{~ostringstream~流}
	当需要一次打印不同数据类型的数据时,使用~ostringstream~流可以很容易实现。比如,在上一节的例子中,在获取所有单词之后,一次性输出每个单词和他们的长度。
	\vspace{4ex}
	\begin{blueblock}{使用示例:}<2->
		\begin{lstlisting}[moreemph={Array,T,Less,F}]
        ostringstream out;//创建流对象;
        for(auto &i;wds)
           out<<i<<":"<<i.lengthe()<<'\n';//处理单词;
        cout<<out.str();
        \end{lstlisting}
	\end{blueblock}
	\begin{redblock}{注意:}<3->
		注意,ostringstream~的另外一个版本的成员函数~str~接受一个~string~类型的参数,用来覆盖原有的数据,例如:\\
		\begin{blueblock}{如下所示:}
			\begin{lstlisting}
                out.str("");//清空原有数据,调用此函数时,out~里面的数据将被清空。
            \end{lstlisting}
		\end{blueblock}
	\end{redblock}
\end{frame}

\begin{frame}[c]{~}
	\begin{center}
		\huge{本章结束}
	\end{center}
\end{frame}

\end{document}