%%%%%%%%%%%%%%%%%%%%%%%%%%%%%%%%%%%%%%%%%
% Beamer Presentation
% LaTeX Template
% Version 1.0 (10/11/12)
%
% This template has been downloaded from:
% http://www.LaTeXTemplates.com
%
% License:
% CC BY-NC-SA 3.0 (http://creativecommons.org/licenses/by-nc-sa/3.0/)
%
%%%%%%%%%%%%%%%%%%%%%%%%%%%%%%%%%%%%%%%%%

%----------------------------------------------------------------------------------------
%	PACKAGES AND THEMES
%----------------------------------------------------------------------------------------

\documentclass[aspectratio=43,UTF8,10pt,t]{ctexbeamer}

\mode<presentation> {
\usetheme{Madrid}
\setbeamertemplate{footline}[frame number] % To remove the footer line in all slides
\setbeamercolor{page number in head/foot}{fg=blue}
\setbeamertemplate{navigation symbols}{} % To remove the navigation symbols from the bottom of all slides
}

% User Defined Block %%%%%%%%%%%%%%%%%%%%%%%%%%%%%%%%%%%%%%%%%%%%%%%%%%%%%%%%
\usepackage{setspace}
\definecolor{hanblue}{rgb}{0.27, 0.42, 0.81}
\definecolor{indiagreen}{rgb}{0.07, 0.53, 0.03}
\definecolor{indianred}{rgb}{0.8, 0.36, 0.36}
\definecolor{indianyellow}{rgb}{0.89, 0.66, 0.34}
\definecolor{babypink}{rgb}{0.96, 0.76, 0.76}
\definecolor{ao(english)}{rgb}{0.0, 0.5, 0.0}
\setbeamerfont{block title}{size=\normalsize}
\setbeamerfont{block body}{size=\small}
\newenvironment<>{blueblock}[1]{%
  \setbeamercolor{block title}{fg=white,bg=hanblue}%
  \begin{block}#2{#1}}{\end{block}}
\newenvironment<>{greenblock}[1]{%
  \setstretch{1.3}\setbeamercolor{block title}{fg=white,bg=indiagreen}%
  \begin{block}#2{#1}}{\end{block}}
\newenvironment<>{redblock}[1]{%
  \setstretch{1.3}\setbeamercolor{block title}{fg=white,bg=indianred}%
  \begin{block}#2{#1}}{\end{block}}
\newenvironment<>{yellowblock}[1]{%
  \setstretch{1.3}\setbeamercolor{block title}{fg=white,bg=indianyellow}%
  \begin{block}#2{#1}}{\end{block}}

%----------------------------------------------------------------------------------------
%	PACKAGES
%----------------------------------------------------------------------------------------
\usepackage{graphicx} % Allows including images
%\usepackage{tikz}
%\usetikzlibrary{shapes.geometric, arrows}
\usepackage{listings}
\lstset{language=C++,
    columns=flexible,
    basicstyle=\footnotesize\ttfamily,                                      % 设定代码字体、大小
    %numbers=left,xleftmargin=2em,framexleftmargin=2em,                   % 在左侧显示行号
    %numberstyle=\color{darkgray},                                        % 设定行号格式
    keywordstyle=\color{blue},                                            % 设定关键字格式
    commentstyle=\color{ao(english)},                                     % 设置代码注释的格式
    stringstyle=\color{brown},                                            % 设置字符串格式
    %showstringspaces=false,                                              % 控制是否显示空格
	%frame=lines,                                                         % 控制外框
    breaklines,                                                           % 控制是否折行
    postbreak=\space,                                                     % 控制折行后显示的标识字符
    breakindent=5pt,                                                      % 控制折行后缩进数量
    emph={size\_t,array,deque,list,map,queue,set,stack,vector,string,pair,tuple}, % 非内置类型
    emphstyle={\color{teal}},
    escapeinside={(*@}{@*)},
}

%----------------------------------------------------------------------------------------
%	TITLE PAGE
%----------------------------------------------------------------------------------------

\title[\textit{C++程序设计:第七章}]{第七章~模板与泛型编程} % The short title appears at the bottom of every slide, the full title is only on the title page

%\author[李长河]{李长河} % Your name
%\institute[CUG] % Your institution as it will appear on the bottom of every slide, may be shorthand to save space
%{
%中国地质大学(武汉)\\ % Your institution for the title page
%\medskip
%\textit{lichanghe@cug.edu.cn} % Your email address
%}
\date{} % Date, can be changed to a custom date

\begin{document}

%----------------------------------------------------------------------------------------
%	TIKZ FLOWCHART
%----------------------------------------------------------------------------------------
%\tikzstyle{startstop} = [rectangle, rounded corners, minimum width=2cm, minimum height=0.5cm, text centered, draw=black, fill=red!30, font=\tiny]
%\tikzstyle{io} = [trapezium, trapezium left angle=70, trapezium right angle=110, minimum width=0cm, minimum height=0cm, text centered, draw=black, fill=blue!30, font=\tiny]
%\tikzstyle{process} = [rectangle, minimum width=2.5cm, minimum height=1.5cm, text centered, draw=black, fill=orange!30, font=\tiny, text width=2cm]
%\tikzstyle{decision} = [diamond, minimum width=2.5cm, minimum height=2cm, text centered, draw=black, fill=green!30, font=\tiny, text width=1.8cm, aspect=1.1]

\begin{frame}
\titlepage % Print the title page as the first slide
\end{frame}

\begin{frame}{目录}
\tableofcontents
\end{frame}

%----------------------------------------------------------------------------------------
%	PRESENTATION SLIDES
%----------------------------------------------------------------------------------------

%--------------------

\begin{frame}[fragile,c]{~} % Table of contents slide, comment this block out to remove it

\begin{block}{学习目标}
\begin{enumerate}
  \item 掌握模板的定义和基本使用方法,包括函数模板和类模板;
  \item 学会运用模板实现泛型编程;
  \item 掌握常用排序算法和二分查找算法。
\end{enumerate}
\end{block}

% ------功能模块说明,请注释掉-------
%\begin{columns}[t]
%\column{0.15\textwidth}
%\begin{block}{概念}
%\end{block}
%\column{0.15\textwidth}
%\begin{blueblock}{代码}
%\end{blueblock}
%\column{0.15\textwidth}
%\begin{yellowblock}{说明}
%\end{yellowblock}
%\column{0.15\textwidth}
%\begin{greenblock}{问题/答案}
%\end{greenblock}
%\column{0.15\textwidth}
%\begin{redblock}{注意}
%\end{redblock}
%\end{columns}
% ------功能模块说明,请注释掉-------

\end{frame}

%--------------------

%#####################################
\section{函数模板}
%#####################################

\begin{frame}[fragile]{7.1~函数模板}

\begin{block}{泛型编程}
泛型编程是指一种采用与数据类型无关的方式编写代码的方法,是\alert{代码重用}的重要手段。\\
~\\
模板是泛型编程的基础,它将\alert{数据类型参数化},为数据结构和算法的抽象提供\alert{通用的代码}解决方案
\end{block}

\vspace{1mm}

\uncover<2->{请观察下面两组代码:}

\vspace{-4mm}

\begin{columns}[t]

\column{0.65\textwidth}
\begin{blueblock}<2->{\texttt{getMax}函数定义一}
\vspace{-1.5mm}\begin{lstlisting}
const int & getMax(const int &a, const int &b){
    return a>b ? a : b;
}
\end{lstlisting}\vspace{-1.5mm}
\end{blueblock}
\begin{blueblock}<2->{\texttt{getMax}函数定义二}
\vspace{-1.5mm}\begin{lstlisting}
const string & getMax(const string &a, const string &b){
    return a>b ? a : b;
}
\end{lstlisting}\vspace{-1.5mm}
\end{blueblock}

\column{0.3\textwidth}
\begin{greenblock}<3->{问题}
这两个定义有什么相同点?
\end{greenblock}

\end{columns}

\end{frame}

%-------------------------------------
\subsection{定义函数模板}
%-------------------------------------

\begin{frame}[fragile]{7.1.1~定义函数模板}

定义\alert{函数模板}来实现\alert{一类}函数的\alert{通用}代码解决方案:

\vspace{-4mm}

\begin{columns}[t]

\column{0.65\textwidth}
\begin{blueblock}{\texttt{getMax}函数模板定义}
\begin{lstlisting}[moreemph={T}]
template <typename T>
const T & getMax(const T &a, const T &b){
    return a>b ? a : b;
}
\end{lstlisting}
\end{blueblock}

\column{0.3\textwidth}
\begin{yellowblock}<2->{说明}
$\bullet$ 模板的定义以关键字~\alert{\texttt{template}}~开始\\
$\bullet$ 模板参数列表放在一对\alert{尖括号}里面\\
$\bullet$ 每一个参数前面用\\\alert{\texttt{typename}}~或~\alert{\texttt{class}}~声明\\
$\bullet$ 列表含有多个模板参数则参数之间用\alert{逗号}分开
\end{yellowblock}
\begin{redblock}<3->{注意}
模板的声明和定义应放在同一个头文件里
\end{redblock}

\end{columns}
\end{frame}

%-------------------------------------
\subsection{实例化函数模板}
%-------------------------------------

\begin{frame}[fragile]{7.1.2~实例化函数模板}

实例化模板函数时,使用者需要提供具体的数据类型或值:

\vspace{-4mm}

\begin{columns}[t]

\column{0.65\textwidth}
\begin{blueblock}<2->{实例化方法一}
\begin{lstlisting}
cout << getMax(1.0, 2.5) << endl; // T为double
\end{lstlisting}
\end{blueblock}
\begin{blueblock}<3->{实例化方法二}
\begin{lstlisting}
cout << getMax<double>(1.0, 2.5) << endl; // 显式指明T为 double
cout << getMax<string>("Hi ", "C++") << endl; // 显式指明T为 string
\end{lstlisting}
\end{blueblock}

\column{0.3\textwidth}
\begin{yellowblock}<2->{说明}
编译器在编译的过程中,利用实参来推断模板参数的类型
\end{yellowblock}
\begin{yellowblock}<3->{说明}
用户显式地指明模板参数的类型
\end{yellowblock}

\end{columns}

\end{frame}

%-----------------

\begin{frame}[fragile]{7.1.2~实例化函数模板\normalsize{~---~为类类型添加模板支持}}

当模板函数的实参为类类型时,需要为类对象添加模板使用到的相关操作:

\vspace{-4mm}

\begin{columns}[t]
\column{0.65\textwidth}
\begin{blueblock}<2->{示例代码}
\begin{lstlisting}[moreemph={Fraction}]
Fraction a(3,4),b(2,5);
cout << getMax(a, b) << endl; // T为Fraction类型
\end{lstlisting}
\end{blueblock}
\column{0.3\textwidth}
\begin{greenblock}<3->{问题}
在编译上面代码时提示编译错误,原因可能是什么?
\end{greenblock}
\begin{greenblock}<4->{答案}
在\texttt{getMax}模板内部用到了关系\texttt{>}运算,但\texttt{Fraction}类不支持关系\texttt{>}运算
\end{greenblock}
\end{columns}

\end{frame}

%-----------------

\begin{frame}[fragile]{7.1.2~实例化函数模板\normalsize{~---~为类类型添加模板支持}}

给\texttt{Fraction}类型添加关系\texttt{>}运算支持:

\vspace{-4mm}

\begin{columns}[t]
\column{0.65\textwidth}
\begin{blueblock}{\texttt{Fraction}类~关系\texttt{>}运算~声明及定义}
\begin{lstlisting}[moreemph={Fraction}]
class Fraction{
    // 将关系>运算声明为Fraction类的友元
    friend bool operator>(const Fraction &lhs, const Fraction &rhs);
    // 其它成员与之前一致
    ...
};

bool operator>(const Fraction &lhs, const Fraction &rhs){
    return lhs.m_numerator*rhs.m_denominator > lhs.m_denominator*rhs.m_numerator;
}
\end{lstlisting}
\end{blueblock}
\column{0.3\textwidth}
\begin{yellowblock}{说明}
根据运算符重载的原则将关系运算符函数\texttt{operator>}作为\texttt{Fraction}类的辅助函数,并将其声明为\texttt{Fraction}类的友元
\end{yellowblock}
\end{columns}

\end{frame}

%-------------------------------------
\subsection{模板参数类型}
%-------------------------------------

\begin{frame}[fragile]{7.1.3~模板参数类型}

以下两组代码中,模板参数有什么区别?

\vspace{-4mm}

\begin{columns}[t]
\column{0.65\textwidth}
\begin{blueblock}{\texttt{foo}函数定义}
\begin{lstlisting}[moreemph={T,U}]
template <typename T, typename U>
T foo(const T &t, const U &u) {
    return T(t);
}
\end{lstlisting}
\end{blueblock}
\begin{blueblock}{\texttt{maxElem}函数定义}
\begin{lstlisting}[moreemph={T}]
template<typename T, int size>
const T& maxElem(T(&arr)[size]) {
    T *p = &arr[0];
    for (auto i = 0; i < size; ++i)
        if (*p < arr[i])
            p = &arr[i];
    return *p;
}
\end{lstlisting}
\end{blueblock}
\column{0.3\textwidth}
\begin{block}<2->{类型参数}
作为\alert{类型说明符},指定函数的返回值类型、形参类型以及函数体内对象的类型等
\end{block}
\begin{block}<3->{非类型参数}
代表一个值,当编译器实例化该模板时必须要为其提供一个\alert{常量}表达式
\end{block}
\begin{yellowblock}<4->{说明}
\texttt{maxElem}函数模板中的函数形参\texttt{arr}为一个指向含有\texttt{size}个\texttt{T}类型数据元素数组的引用
\end{yellowblock}
\end{columns}

\end{frame}

%-------------------

\begin{frame}[fragile]{7.1.3~模板参数类型}

调用\texttt{maxElem}函数:

\vspace{-4mm}

\begin{columns}[t]

\column{0.65\textwidth}
\begin{blueblock}{\texttt{maxElem}函数模板实例化}
\begin{lstlisting}
int arr[10] = {1,8,5,3};
int x = maxElem(arr);
// 或者显式调用 maxElem<int,10>(arr);
\end{lstlisting}
\end{blueblock}
编译器将会生成如下版本的函数:
\begin{blueblock}{}
\vspace{-2.5mm}\begin{lstlisting}
const int& maxElem(int (&arr)[10]);
\end{lstlisting}\vspace{-2mm}
\end{blueblock}

\column{0.3\textwidth}
\begin{greenblock}<2->{问题}
还有什么传递数组参数的方式?
\end{greenblock}
\begin{greenblock}<3->{答案}
还可以通过指针传递数组首地址的方式
\end{greenblock}

\end{columns}

\end{frame}

%------------------

\begin{frame}[fragile]{7.1.3~模板参数类型\normalsize{~---~模板重载与特化}}

如果前面定义的\texttt{getMax}函数模板在调用过程中的实参为指针类型:

\vspace{-4mm}

\begin{columns}[t]

\column{0.65\textwidth}
\begin{blueblock}{\texttt{getMax}函数调用一}
\begin{lstlisting}[moreemph={T}]
int a = 1, b = 2;
getMax(&a, &b);
\end{lstlisting}
\end{blueblock}
\begin{blueblock}{\texttt{getMax}定义一}
\begin{lstlisting}[moreemph={T}]
template <typename T>
const T & getMax(const T & a, const T & b){
    return a > b ? a : b;
}
\end{lstlisting}
\end{blueblock}

\column{0.3\textwidth}
\begin{yellowblock}{说明}
需要返回两个指针所指向的对象的比较结果
\end{yellowblock}
\begin{greenblock}<2->{问题}
\texttt{getMax}函数模板的定义还能否满足要求?
\end{greenblock}
\begin{greenblock}<3->{答案}
不能。编译器推演出的参数\texttt{T}为\texttt{int*},函数体里面的操作变成了两个指针对象的比较
\end{greenblock}

\end{columns}

\end{frame}

%-----------------

\begin{frame}[fragile]{7.1.3~模板参数类型\normalsize{~---~模板重载与特化}}

为此,需要\alert{重载}一个\texttt{getMax}模板函数:

\vspace{-4mm}

\begin{columns}[t]

\column{0.65\textwidth}
\begin{blueblock}{\texttt{getMax}函数调用二}
\begin{lstlisting}[moreemph={T}]
int a = 1, b = 2;
getMax(&a, &b);
\end{lstlisting}
\end{blueblock}
\begin{blueblock}{\texttt{getMax}函数模板重载}
\begin{lstlisting}[moreemph={T}]
template <typename T>
T* const & getMax( T* const &a, T* const &b){
    return *a>*b ? a : b;
}
\end{lstlisting}
\end{blueblock}

\column{0.3\textwidth}
\begin{yellowblock}{说明}
模板实参\texttt{T}的类型为\texttt{int},\texttt{*a}和\texttt{*b}指向的是\texttt{int}对象,函数体里面的操作是两个\texttt{int}对象的比较
\end{yellowblock}

\end{columns}

\end{frame}

%-----------------

\begin{frame}[fragile]{7.1.3~模板参数类型\normalsize{~---~模板重载与特化}}

进一步,如果调用的实参是指向字符串的指针:

\vspace{-4mm}

\begin{columns}[t]

\column{0.65\textwidth}
\begin{blueblock}{\texttt{getMax}函数调用三}
\begin{lstlisting}[moreemph={T}]
const char *a = "Hi", *b = "C++";
cout << getMax(a, b) << endl;
\end{lstlisting}
\end{blueblock}
\begin{blueblock}{\texttt{getMax}函数定义二}
\vspace{-2.5mm}\begin{lstlisting}[moreemph={T}]
template <typename T>
const T & getMax(const T & a, const T & b){
    return a > b ? a : b;
}

template <typename T>
T* const & getMax( T* const &a, T* const &b){
    return *a>*b ? a : b;
}
\end{lstlisting}\vspace{-2mm}
\end{blueblock}

\column{0.3\textwidth}
\begin{yellowblock}{说明}
需要返回指向字符串值较大的字符指针
\end{yellowblock}
\begin{greenblock}<2->{问题}
现有的两个\texttt{getMax}函数的定义还能否满足要求?
\end{greenblock}
\begin{greenblock}<3->{答案}
不能。\texttt{*a}和\texttt{*b}指向的是单个字符,函数体里面的操作变成了两个字符的比较
\end{greenblock}

\end{columns}

\end{frame}

%-----------------

\begin{frame}[fragile]{7.1.3~模板参数类型\normalsize{~---~模板重载与特化}}

为此,需要\alert{特例化}一个\texttt{getMax}模板函数:

\vspace{-4mm}

\begin{columns}[t]

\column{0.65\textwidth}
\begin{blueblock}{\texttt{getMax}函数调用三}
\begin{lstlisting}[moreemph={T}]
const char *a = "Hi", *b = "C++";
cout << getMax(a, b) << endl;
\end{lstlisting}
\end{blueblock}
\begin{blueblock}{\texttt{getMax}函数模板特化}
\begin{lstlisting}[moreemph={T}]
template <>
const char* const & getMax(const char* const &a, const char* const &b){
    return strcmp(a,b) > 0 ? a : b;
}
\end{lstlisting}
\end{blueblock}


\column{0.3\textwidth}
\begin{yellowblock}{说明}
模板参数列表为空,表明将显式提供所有模板实参
\end{yellowblock}
\vspace{-2mm}
\begin{yellowblock}{说明}
\texttt{T}被推断为\texttt{const char*},\texttt{a}和\texttt{b}分别为一个指向\texttt{const char}对象的\texttt{const}指针的引用,函数是对两个字符串值的比较
\end{yellowblock}
\vspace{-2mm}
\begin{redblock}<2->{注意}
一个特例化的函数模板本质上是一个实例,而非函数名的一个重载版本
\end{redblock}

\end{columns}

\end{frame}

%-----------------

\begin{frame}[fragile]{7.1.3~模板参数类型\normalsize{~---~模板重载与特化}}

还可以通过模板特化改善算法:

\vspace{-4mm}

\begin{columns}[t]

\column{0.65\textwidth}
\begin{blueblock}<2->{\texttt{Swap}函数模板定义}
\vspace{-2.5mm}\begin{lstlisting}[moreemph={T}]
template<typename T>
void Swap(T &a, T &b) {
    T c(a); // 复制构造对象 c
    a = b;
    b = c;
}
\end{lstlisting}\vspace{-2mm}
\end{blueblock}

\column{0.3\textwidth}
\begin{yellowblock}<2->{说明}
需要构造一个辅助的局部对象\texttt{c},才能完成\texttt{a}和\texttt{b}的交换
\end{yellowblock}

\end{columns}

\vspace{4mm}

\uncover<3->{如果\texttt{T}是\texttt{int},可以利用模板特化做出优化:}

\vspace{-4mm}

\begin{columns}[t]

\column{0.65\textwidth}
\begin{blueblock}<4->{\texttt{Swap}函数模板特化}
\vspace{-2.5mm}\begin{lstlisting}[moreemph={T}]
template<>
void Swap(int &a, int &b)
    a ^= b;
    b ^= a;
    a ^= b;
}
\end{lstlisting}\vspace{-2mm}
\end{blueblock}

\column{0.3\textwidth}
\begin{yellowblock}<4->{说明}
利用异或操作完成两个整数的交换,没有创建辅助对象,没有产生构造和析构行为,提高了执行效率
\end{yellowblock}

\end{columns}

\end{frame}

%-----------------

%-------------------------------------
\subsection{类成员模板}
%-------------------------------------

%-----------------
\begin{frame}[fragile]{7.1.4~类成员模板}

类的成员函数也可以定义为函数模板:

\vspace{-4mm}

\begin{columns}[t]

\column{0.65\textwidth}
\begin{blueblock}{类\texttt{X}定义}
\begin{lstlisting}[moreemph={T}]
class X{
    void * m_p = nullptr;
public:
    template <typename T>
    void reset(T *t) { m_p = t; }
};
\end{lstlisting}
\end{blueblock}
\begin{blueblock}<2->{\texttt{reset}函数调用}
\begin{lstlisting}[moreemph={T}]
int i = 0;
double d = 0;
X x;
x.reset(&i); // 或者x.reset<int>(&i);
x.reset(&d); // 或者x.reset<double>(&d);
\end{lstlisting}
\end{blueblock}

\column{0.3\textwidth}
\begin{yellowblock}{说明}
成员函数\texttt{reset}定义为一个函数模板,接受不同类型的指针实参
\end{yellowblock}
\begin{yellowblock}<2->{说明}
$\bullet$ 第一条\texttt{reset}函数调用中\texttt{T}被推断为\texttt{int}类型,\texttt{m\_p}存放整型对象\texttt{i}的地址\\
$\bullet$ 第二条\texttt{reset}函数调用中\texttt{T}被推断为\texttt{double}类型,\texttt{m\_p}存放\texttt{double}类对象\texttt{d}的地址
\end{yellowblock}

\end{columns}

\end{frame}
%-----------------

%-------------------------------------
\subsection{可变参函数模板}
%-------------------------------------

%-----------------
\begin{frame}[fragile]{7.1.5~可变参函数模板}

C++11新标准允许我们使用\alert{数目可变}的模板参数:

\vspace{-4mm}

\begin{columns}[t]

\column{0.65\textwidth}
\begin{blueblock}{\texttt{foo}函数定义}
\begin{lstlisting}[moreemph={T}]
template<typename... Args >
    void foo(Args... args) {
    // 打印参数包args中参数的个数
    cout << sizeof...(args) << endl;
}
\end{lstlisting}
\end{blueblock}
\begin{blueblock}<2->{\texttt{foo}函数调用}
\begin{lstlisting}[moreemph={T}]
foo(); // 输出:0
foo(1,1.5); // 输出:2
foo(1,1.5,"C++"); // 输出:3
\end{lstlisting}
\end{blueblock}

\column{0.3\textwidth}
\begin{yellowblock}{说明}
$\bullet$ 可变数目的参数称为\alert{参数包},用省略号“\alert{\texttt{...}}”表示,可包含0到任意个模板参数\\
$\bullet$ \texttt{foo}函数的形参\texttt{args}为模板参数包类型,接受可变数目的实参
\end{yellowblock}

\end{columns}

\end{frame}

%-----------------

\begin{frame}[fragile]{7.1.5~可变参函数模板\normalsize{~---~包展开}}

可以通过\alert{递归}的方式展开函数模板参数包:

\vspace{-4mm}

\begin{columns}[t]

\column{0.65\textwidth}
\begin{blueblock}{\texttt{print}函数定义}
\vspace{-2mm}\begin{lstlisting}[moreemph={T}]
template<typename T, typename... Args>
ostream& print(ostream &os, const T &t, const Args&... rest) {
    os << t << " "; // 打印第一个参数
    return print(os, rest...); // 递归调用
}

template<typename T>
ostream& print(ostream &os, const T &t) {
    return os << t; // 打印最后一个参数
}
\end{lstlisting}\vspace{-2mm}
\end{blueblock}
\begin{blueblock}<2->{\texttt{print}函数调用}
\vspace{-1.5mm}\begin{lstlisting}[moreemph={T}]
print(cout,1, 2.5, "C++") << endl; // 输出1 2.5 C++
\end{lstlisting}\vspace{-1.5mm}
\end{blueblock}

\column{0.3\textwidth}
\begin{yellowblock}{说明}
$\bullet$ 第一次处理参数包中的第一个参数,然后用剩余参数调用自身\\
$\bullet$ 当参数包里面只剩下一个参数时,非可变参模板与可变参模板都匹配,但是非可变参模板更特例化,编译器首选非可变参数版本
\end{yellowblock}
\begin{greenblock}<3->{问题}
\texttt{print}函数的形参是左值引用,如果是右值引用呢?
\end{greenblock}

\end{columns}

\end{frame}

%-----------------

\begin{frame}[fragile]{7.1.5~可变参函数模板\normalsize{~---~转发参数包}}

两个函数的形参都是右值引用,\texttt{forwardValue}函数调用报错,为什么?

\vspace{-4mm}

\begin{columns}[t]

\column{0.65\textwidth}
\begin{blueblock}{\texttt{forwardValue}函数及\texttt{rvalue}函数定义}
\vspace{-2mm}\begin{lstlisting}[moreemph={T}]
void rvalue(int &&val){}

template<typename T>
void forwardValue(T &&val) {
    rvalue(val);
}
\end{lstlisting}\vspace{-2mm}
\end{blueblock}
\begin{blueblock}{\texttt{forwardValue}函数调用}
\vspace{-2mm}\begin{lstlisting}[moreemph={T}]
forwardValue(42); // 错误
\end{lstlisting}\vspace{-2mm}
\end{blueblock}
\begin{blueblock}<2->{\texttt{forwardValue}函数调用细节}
\vspace{-2mm}\begin{lstlisting}[moreemph={T}]
T &&val = 42; // 等同于auto &&val = 42
rvalue(val);
\end{lstlisting}\vspace{-2mm}
\end{blueblock}

\column{0.3\textwidth}
\begin{greenblock}<2->{答案}
$\bullet$ 右值引用声明 \texttt{\&\&}与类型推导结合可以与右值绑定,所以\texttt{val}为右值引用\\
$\bullet$ 临时对象\texttt{42}通过右值引用\texttt{val}引用之后生命期能延续是因为\alert{右值引用\texttt{val}是左值}\\
$\bullet$ \texttt{rvalue}函数只接受右值形参,但\texttt{val}是左值
\end{greenblock}
\begin{greenblock}<3->{问题}
可以让传入\texttt{rvalue}函数的\alert{实参保持原属性吗}?
\end{greenblock}

\end{columns}

\end{frame}

%-----------------

\begin{frame}[fragile]{7.1.5~可变参函数模板\normalsize{~---~转发参数包}}

在C++11新标准下可以利用~\alert{\texttt{std::forward}}~函数实现:

\vspace{-4mm}

\begin{columns}[t]

\column{0.65\textwidth}
\begin{blueblock}{\texttt{std::forward}函数描述性声明}
\vspace{-2mm}\begin{lstlisting}[moreemph={T}]
template<typename T>
T&& forward(T val);
\end{lstlisting}\vspace{-2mm}
\end{blueblock}
\begin{blueblock}<2->{\texttt{forwardValue}函数定义二}
\vspace{-2mm}\begin{lstlisting}[moreemph={T}]
template<typename T>
void forwardValue(T &&val) {
    rvalue(std::forward<T>(val));
}
\end{lstlisting}\vspace{-2mm}
\end{blueblock}
\begin{blueblock}<2->{\texttt{forwardValue}函数调用二}
\vspace{-2mm}\begin{lstlisting}[moreemph={T}]
forwardValue(42); // 正确
int a = 42;
forwardValue(a); // 正确
\end{lstlisting}\vspace{-2mm}
\end{blueblock}

\column{0.3\textwidth}
\begin{yellowblock}<3->{说明}
当传入\texttt{forwardValue}的实参为\texttt{42}是右值,\texttt{T}被推断为非引用类型,\texttt{forward<T>}将返回右值
\end{yellowblock}
\vspace{-2mm}
\begin{yellowblock}<4->{说明}
当传入\texttt{forwardValue}的实参为\texttt{a}是左值,\texttt{T}被推断为左值引用类型,此时\texttt{forward<T>}将返回左值
\end{yellowblock}
\vspace{-2mm}
\begin{redblock}<5->{注意}
在C++11新标准下,\\
\texttt{\&\&\&} 折叠为 \texttt{\&}
\end{redblock}

\end{columns}

\end{frame}

%-----------------

\begin{frame}[fragile]{7.1.5~可变参函数模板\normalsize{~---~转发参数包}}

可以组合使用可变参模板与\texttt{forward}函数,实现\alert{参数包完美转发}:

\vspace{-4mm}

\begin{columns}[t]

\column{0.65\textwidth}
\begin{blueblock}<2->{\texttt{fun}函数定义二和\texttt{foo}函数定义二}
\vspace{-2mm}\begin{lstlisting}[moreemph={Args}]
void foo(const string &s, int &&i) {
    cout << s << i << endl;
}

template<typename... Args>
void fun(Args&&... args) {
    foo(std::forward<Args>(args)...);
}
\end{lstlisting}\vspace{-2mm}
\end{blueblock}
\uncover<3->{当进行如下调用:}
\begin{blueblock}<3->{\texttt{fun}函数调用}
\vspace{-2.5mm}\begin{lstlisting}[moreemph={T}]
fun("abc", 42);
\end{lstlisting}\vspace{-2mm}
\end{blueblock}

\column{0.3\textwidth}
\begin{yellowblock}<2->{说明}
\texttt{std::forward<Args>(args)...}\\
相当于:\\
\texttt{std::forward<$T_i$>($t_i$)}\\
其中$T_i$为参数包中第$i$个参数$t_i$的类型
\end{yellowblock}

\end{columns}

\vspace{3mm}

\uncover<3->{\texttt{foo}函数的实参将扩展为:}

\begin{blueblock}<3->{}
\vspace{-2.5mm}\begin{lstlisting}[moreemph={T}]
std::forward<const char * >("abc"), std::forward<int>(42)
\end{lstlisting}\vspace{-2mm}
\end{blueblock}

\end{frame}

%-----------------

%###############################################################################
\section{类模板}
%###############################################################################

%-----------------

\begin{frame}[fragile]{7.2~类模板}

类似函数模板,可以定义一个类模板用来生成具有\alert{相同结构}的一族类实例:

\vspace{-4mm}

\begin{columns}[t]

\column{0.65\textwidth}
\begin{blueblock}{\texttt{Array}类模板定义}
\begin{lstlisting}[moreemph={T}]
template<typename T, size_t N>
class Array {
    T m_ele[N];
public:
    Array() {}
    Array(const std::initializer_list<T> &);
    T& operator[](size_t i);
    constexpr size_t size() { return N; }
};
\end{lstlisting}
\end{blueblock}

\column{0.3\textwidth}
\begin{yellowblock}{说明}
$\bullet$ 类型参数\texttt{T}和非类型参数\texttt{N},分别用来表示元素的类型和元素的数目\\
$\bullet$ \alert{\texttt{initializer\_list}}~类型是C++11标准库提供的新类型,支持具有相同类型但数量未知的列表类型
\end{yellowblock}

\end{columns}

\end{frame}

%-----------------

%-------------------------------------
\subsection{成员函数定义}
%-------------------------------------

%-----------------

\begin{frame}[fragile]{7.2.1~成员函数定义}

与普通类相同,既可以在类的内部,也可以在\alert{类的外部}定义\alert{类模板成员函数}:

\vspace{-4mm}

\begin{columns}[t]

\column{0.65\textwidth}
\begin{blueblock}{\texttt{Array}类模板~构造函数~类外定义}
\begin{lstlisting}[moreemph={T,Array,initializer}]
template<typename T, size_t N>
Array<T, N>::Array(const std::initializer_list<T> &l):m_ele{T()}{
    size_t m = l.size() < N ? l.size() : N;
    for (size_t i = 0; i < m; ++i) {
        m_ele[i] = *(l.begin() + i);
    }
}
\end{lstlisting}
\end{blueblock}

\column{0.3\textwidth}
\begin{yellowblock}{说明}
$\bullet$ 必须以关键字\texttt{template}开始,后接\alert{与类模板相同}的模板参数列表\\
$\bullet$ 紧随类名后面的参数列表代表一个实例化的实参列表,每个参数不需要\texttt{typename}或\texttt{class}说明符
\end{yellowblock}

\end{columns}

\end{frame}

%-----------------

\begin{frame}[fragile]{7.2.1~成员函数定义}

与普通类相同,既可以在类的内部,也可以在\alert{类的外部}定义\alert{类模板成员函数}:

\vspace{-4mm}

\begin{columns}[t]

\column{0.65\textwidth}
\begin{blueblock}{\texttt{Array}类模板~构造函数~类外定义}
\begin{lstlisting}[moreemph={T,Array,initializer}]
template<typename T, size_t N>
Array<T, N>::Array(const std::initializer_list<T> &l):m_ele{T()}{
    size_t m = l.size() < N ? l.size() : N;
    for (size_t i = 0; i < m; ++i) {
        m_ele[i] = *(l.begin() + i);
    }
}
\end{lstlisting}
\end{blueblock}
\begin{blueblock}<2->{\texttt{Array}类模板~\texttt{[]}运算符函数~类外定义}
\begin{lstlisting}[moreemph={T,Array}]
template<typename T, size_t N>
T& Array<T, N>::operator[](size_t i) {
    return m_ele[i];
}
\end{lstlisting}
\end{blueblock}

\column{0.3\textwidth}
\begin{yellowblock}{说明}
$\bullet$ \texttt{m\_ele}中的每一个元素用T类型的默认初始化方式(\texttt{T()})初始化\\
$\bullet$ 将形参\texttt{l}中的元素依次复制到类模板数组成员\texttt{m\_ele}中相应的位置\\
$\bullet$ \texttt{l.begin}返回列表\texttt{l}中第一个元素的迭代器
\end{yellowblock}
\begin{yellowblock}<2->{说明}
类模板的下标运算符函数返回数组\texttt{m\_ele}中第\texttt{i}个元素的引用
\end{yellowblock}

\end{columns}

\end{frame}

%-----------------

%-------------------------------------
\subsection{实例化类模板}
%-------------------------------------

%-----------------

\begin{frame}[fragile]{7.2.2~实例化类模板}

当使用一个类模板时,我们需要\alert{显式}提供模板参数信息,即\alert{模板实参列表}:

\vspace{-4mm}

\begin{columns}[t]

\column{0.65\textwidth}
\begin{blueblock}{实例化\texttt{Array}类模板}
\begin{lstlisting}[moreemph={Array}]
Array<char, 5> a; //创建一个Array<char, 5>类型对象 a
Array<int, 5> b = {1,2,3}; //创建一个Array<int, 5>类型对象 b
\end{lstlisting}
\end{blueblock}
\uncover<2->{下面代码逐个输出对象\texttt{b}的每一个元素:}
\begin{blueblock}<2->{}
\begin{lstlisting}[moreemph={Array}]
for (int i = 0; i < b.size(); ++i)
    cout << b[i] << " ";
\end{lstlisting}
\end{blueblock}
\uncover<2->{输出结果为:\texttt{1 2 3 0 0}}

\column{0.3\textwidth}
\begin{yellowblock}{说明}
创建对象\texttt{b}时,将执行具有形参的构造函数,其形参\texttt{l}接受初始化列表 \texttt{\{1,2,3\}},其余元素具有默认值\texttt{0}
\end{yellowblock}


\end{columns}

\end{frame}

%-----------------

%-------------------------------------
\subsection{默认模板参数}
%-------------------------------------

%-----------------

\begin{frame}[fragile]{7.2.3~默认模板参数}

\alert{函数参数}可以具有默认值,\alert{模板参数}同样也可以有默认值:

\vspace{-4mm}

\begin{columns}[t]

\column{0.65\textwidth}
\begin{blueblock}{\texttt{Array}类模板定义二}
\vspace{-1mm}\begin{lstlisting}[moreemph={Array,T,Less,F}]
template<typename T = int, size_t N = 10>
class Array {
    // 其它成员保持不变
};
\end{lstlisting}\vspace{-1mm}
\end{blueblock}
\begin{blueblock}<2->{实例化\texttt{Array}类模板二}
\begin{lstlisting}[moreemph={Array}]
Array<> a = { 'A' };
cout << a.size() << " " << a[0] << endl;
\end{lstlisting}
\end{blueblock}
\uncover<2->{输出结果为:\texttt{10 65}}

\column{0.3\textwidth}
\begin{yellowblock}{说明}
$\bullet$ \alert{类模板参数}~\texttt{T}具有默认类型\texttt{int}\\
$\bullet$ \alert{类模板参数}~\texttt{N}具有默认值\texttt{10}
\end{yellowblock}
\begin{yellowblock}<2->{说明}
$\bullet$ \texttt{a}的元素数目为默认值\texttt{10}\\
$\bullet$ \texttt{a[0]}的类型为\texttt{int},字符\texttt{'A'}转换为\texttt{65}
\end{yellowblock}

\end{columns}

\end{frame}

%-----------------

\begin{frame}[fragile]{7.2.3~默认模板参数}

和\alert{类模板参数}一样,\alert{函数模板参数}也可以有默认值:

\vspace{-4mm}

\begin{columns}[t]

\column{0.65\textwidth}
\begin{blueblock}{\texttt{Array}类模板定义三}
\vspace{-1mm}\begin{lstlisting}[moreemph={Array,T,Less,F}]
template<typename T, size_t N>
class Array {
    // 其它成员保持不变
public:
    template<typename F = Less<T>>
    void sort(F f = F());
};
\end{lstlisting}\vspace{-1mm}
\end{blueblock}
\begin{blueblock}<2->{\texttt{Less}类模板定义}
\vspace{-1mm}\begin{lstlisting}[moreemph={Less,T}]
template<typename T>
struct Less{
    bool operator()(const T &a, const T &b) {
        return a < b;
    }
};
\end{lstlisting}\vspace{-1mm}
\end{blueblock}

\column{0.3\textwidth}
\begin{yellowblock}{说明}
$\bullet$ 新增了一个成员函数模板\texttt{sort},用来对数组进行排序\\
$\bullet$ \texttt{sort}的\alert{函数模板参数}~\texttt{F}具有默认值\texttt{Less<T>}
\end{yellowblock}
\begin{yellowblock}<2->{说明}
类模板\texttt{Less<T>}具有一个模板参数\texttt{T},且只有一个函数调用运算符,该成员函数带有两个形参,用来比较两个形参的大小,返回值类型为\texttt{bool}
\end{yellowblock}

\end{columns}

\end{frame}

%-----------------

\begin{frame}[fragile]{7.2.3~默认模板参数}

和\alert{类模板参数}一样,\alert{函数模板参数}也可以有默认值:

\vspace{-4mm}

\begin{columns}[t]

\column{0.65\textwidth}
\begin{blueblock}{\texttt{Array}类模板定义三}
\vspace{-1mm}\begin{lstlisting}[moreemph={Array,T,Less,F}]
template<typename T, size_t N = 10>
class Array {
    // 其它成员保持不变
public:
    template<typename F = Less<T>>
    void sort(F f = F());
};
\end{lstlisting}\vspace{-1mm}
\end{blueblock}

\column{0.3\textwidth}
\begin{yellowblock}{说明}
$\bullet$ \texttt{sort}的\alert{函数参数}~\texttt{f}也有默认值,即\texttt{F}类的一个函数对象,代表默认比较方式为\texttt{Less}
\end{yellowblock}
\begin{greenblock}<2->{问题}
理清~\alert{函数参数}、\alert{模板参数}、\alert{类模板参数}、\alert{函数模板参数}~的概念
\end{greenblock}

\end{columns}

\end{frame}

%-----------------

%###############################################################################
\section{排序与查找}
%###############################################################################

%-------------------------------------
\subsection{排序算法}
%-------------------------------------

%-----------------

\begin{frame}[fragile]{7.3.1~排序算法}

排序是数据处理的最基本任务,目的是按照某种规则将一组无序数据重新排列,使之有序。下面将给\texttt{Array}模板类增加三种最常用的排序算法:选择排序、插入排序和冒泡排序

\vspace{-4mm}

\begin{columns}[t]

\column{0.65\textwidth}
\begin{blueblock}<2->{\texttt{Array}类模板定义四}
\vspace{-3mm}
\begin{lstlisting}[moreemph={Array,T}]
template<typename T, size_t N>
class Array {
    //其它成员保持不变
public:
    template<typename F = Less<T> >
    void selectionSort(F f = F()); //选择排序
    template<typename F = Less<T> >
    void insertionSort(F f = F()); //插入排序
    template<typename F = Less<T> >
    void bubbleSort(F f = F()); //冒泡排序
private:
    void swap(int i, int j){
        T t = m_ele[i];
        m_ele[i] = m_ele[j];
        m_ele[j] = t;
    }
};
\end{lstlisting}
\vspace{-3mm}
\end{blueblock}

\column{0.3\textwidth}
\begin{yellowblock}<2->{说明}
成员\texttt{swap}函数用来交换两个元素的位置,它仅在\texttt{Array}类内部使用,因此它的访问属性为\texttt{private}
\end{yellowblock}

\end{columns}

\end{frame}

%-----------------

\begin{frame}[fragile]{7.3.1 排序算法\normalsize{~---~选择排序}}

每次在待排序元素中选择最小的一个,换放到已排序数列后面

\vspace{1mm}

\begin{center}
\includegraphics[width=0.8\textwidth]{selection_sort_1}\\\vspace{3mm}
\uncover<2->{\includegraphics[width=0.8\textwidth]{selection_sort_2}}\\\vspace{-8.9mm}
\uncover<3->{\includegraphics[width=0.8\textwidth]{selection_sort_3}}\\
\uncover<4->{\includegraphics[width=0.8\textwidth]{selection_sort_4}}\\\vspace{-8.9mm}
\uncover<5->{\includegraphics[width=0.8\textwidth]{selection_sort_5}}\\
\uncover<6->{\includegraphics[width=0.8\textwidth]{selection_sort_6}}\\\vspace{-8.9mm}
\uncover<7->{\includegraphics[width=0.8\textwidth]{selection_sort_7}}\\
\uncover<8->{\includegraphics[width=0.8\textwidth]{selection_sort_8}}\\\vspace{-8.9mm}
\uncover<9->{\includegraphics[width=0.8\textwidth]{selection_sort_9}}\\\vspace{3mm}
\uncover<10->{\includegraphics[width=0.8\textwidth]{selection_sort_10}}
\end{center}

\end{frame}

%-----------------

\begin{frame}[fragile]{7.3.1 排序算法\normalsize{~---~选择排序}}

选择排序算法的实现如下:

\vspace{-4mm}

\begin{columns}[t]

\column{0.65\textwidth}
\begin{blueblock}{\texttt{Array}成员函数\texttt{selectionSort}定义}
\begin{lstlisting}[moreemph={Array,T,F}]
template<typename T, size_t N>
template<typename F >
void Array<T, N>::selectionSort(F f) {
    for (int i = 0; i < N - 1; ++i){
        int min = i; // 记录待排序数据中最小元素位置
        for (int j = i + 1; j < N; ++j) {
            if (f(m_ele[j], m_ele[min]))
                min = j; //更新最小元素位置
        }
        swap(i, min); //把最小元素放到位置i
    }
}
\end{lstlisting}
\end{blueblock}

\column{0.3\textwidth}
\begin{yellowblock}{说明}
\texttt{if}语句里的条件表达式将调用函数对象\texttt{f}(\texttt{Less<T>}),检查第一个实参对象是否小于第二个实参对象
\end{yellowblock}

\end{columns}

\end{frame}

%-----------------

\begin{frame}[fragile]{7.3.1 排序算法\normalsize{~---~插入排序}}

将待排序的元素逐个插入已经排好序的元素序列中

\vspace{1mm}

\begin{center}
\includegraphics[width=0.8\textwidth]{insertion_sort_1}\\\vspace{3mm}
\uncover<2->{\includegraphics[width=0.8\textwidth]{insertion_sort_2}}\\
\uncover<3->{\includegraphics[width=0.8\textwidth]{insertion_sort_3}}\\\vspace{-8.73mm}
\uncover<4->{\includegraphics[width=0.8\textwidth]{insertion_sort_4}}\\\vspace{-8.73mm}
\uncover<5->{\includegraphics[width=0.8\textwidth]{insertion_sort_5}}\\
\uncover<6->{\includegraphics[width=0.8\textwidth]{insertion_sort_6}}\\\vspace{-8.73mm}
\uncover<7->{\includegraphics[width=0.8\textwidth]{insertion_sort_7}}\\\vspace{-8.73mm}
\uncover<8->{\includegraphics[width=0.8\textwidth]{insertion_sort_8}}\\
\uncover<9->{\includegraphics[width=0.8\textwidth]{insertion_sort_9}}\\\vspace{-8.73mm}
\uncover<10->{\includegraphics[width=0.8\textwidth]{insertion_sort_10}}\\\vspace{-8.73mm}
\uncover<11->{\includegraphics[width=0.8\textwidth]{insertion_sort_11}}\\\vspace{3mm}
\uncover<12->{\includegraphics[width=0.8\textwidth]{insertion_sort_12}}\\\vspace{-8.73mm}
\uncover<13->{\includegraphics[width=0.8\textwidth]{insertion_sort_13}}
\end{center}

\end{frame}

%-----------------

\begin{frame}[fragile]{7.3.1 排序算法\normalsize{~---~插入排序}}

插入排序算法的实现如下:

\vspace{-4mm}

\begin{columns}[t]

\column{0.65\textwidth}
\begin{blueblock}{\texttt{Array}成员函数\texttt{insertionSort}定义}
\begin{lstlisting}[moreemph={Array,T,F}]
template<typename T, size_t N>
template<typename F >
void Array<T, N>::insertionSort(F f) {
    for (int i = 1, j; i < N; ++i) {
        T t = m_ele[i]; //待插入元素
        for (j = i; j > 0; --j) { //查找插入位置
            if (f(m_ele[j - 1], t))
                break;
            m_ele[j] = m_ele[j - 1]; //逐个向后移动元素
        }
        m_ele[j] = t; //将待插入元素放到正确位置
    }
}
\end{lstlisting}
\end{blueblock}

\column{0.3\textwidth}

\end{columns}

\end{frame}

%-----------------

\begin{frame}[fragile]{7.3.1 排序算法\normalsize{~---~冒泡排序}}

不断比较相邻的两个元素,如果发现逆序则交换

\vspace{1mm}

\begin{center}
\includegraphics[width=0.8\textwidth]{bubble_sort_1}\\\vspace{3mm}
\uncover<2->{\includegraphics[width=0.8\textwidth]{bubble_sort_2}}\\\vspace{-8.73mm}
\uncover<3->{\includegraphics[width=0.8\textwidth]{bubble_sort_3}}\\
\uncover<4->{\includegraphics[width=0.8\textwidth]{bubble_sort_4}}\\\vspace{-8.73mm}
\uncover<5->{\includegraphics[width=0.8\textwidth]{bubble_sort_5}}\\
\uncover<6->{\includegraphics[width=0.8\textwidth]{bubble_sort_6}}\\\vspace{-8.73mm}
\uncover<7->{\includegraphics[width=0.8\textwidth]{bubble_sort_7}}\\
\uncover<8->{\includegraphics[width=0.8\textwidth]{bubble_sort_8}}\\\vspace{3mm}
\uncover<9->{\includegraphics[width=0.8\textwidth]{bubble_sort_9}}
\end{center}

\end{frame}

%-----------------

\begin{frame}[fragile]{7.3.1 排序算法\normalsize{~---~冒泡排序}}

冒泡排序算法的实现如下:

\vspace{-4mm}

\begin{columns}[t]

\column{0.65\textwidth}
\begin{blueblock}{\texttt{Array}成员函数\texttt{selectionSort}定义}
\begin{lstlisting}[moreemph={Array,T,F}]
template<typename T, size_t N>
template<typename F >
void Array<T, N>::bubbleSort(F f){
    for (int i = N - 1; i >= 0; --i){
        for (int j = 0; j <= i - 1; ++j){
            if (f(m_ele[j + 1], m_ele[j]))
                swap(j, j + 1); //相邻元素交换
        }
    }
}
\end{lstlisting}
\end{blueblock}

\column{0.3\textwidth}

\end{columns}

\end{frame}

%-----------------

%-------------------------------------
\subsection{二分查找算法}
%-------------------------------------

%-----------------

\begin{frame}[fragile]{7.3.2~二分查找算法}

又称折半查找,在有序序列中使用,其基本思想为分而治之

\vspace{1mm}

\begin{center}
\uncover<2->{\includegraphics[width=0.8\textwidth]{binary_search_1}}\\\vspace{3mm}
\uncover<3->{\includegraphics[width=0.8\textwidth]{binary_search_2}}\\\vspace{-8.22mm}
\uncover<4->{\includegraphics[width=0.8\textwidth]{binary_search_3}}\\\vspace{-8.22mm}
\uncover<5->{\includegraphics[width=0.8\textwidth]{binary_search_4}}\\\vspace{-8.22mm}
\uncover<6->{\includegraphics[width=0.8\textwidth]{binary_search_5}}\\\vspace{-8.22mm}
\uncover<7->{\includegraphics[width=0.8\textwidth]{binary_search_6}}\\\vspace{12mm}
\uncover<8->{\includegraphics[width=0.8\textwidth]{binary_search_7}}\\\vspace{3mm}
\uncover<9->{\includegraphics[width=0.8\textwidth]{binary_search_8}}\\\vspace{-8.22mm}
\uncover<10->{\includegraphics[width=0.8\textwidth]{binary_search_9}}\\\vspace{-8.22mm}
\uncover<11->{\includegraphics[width=0.8\textwidth]{binary_search_10}}\\\vspace{-8.22mm}
\uncover<12->{\includegraphics[width=0.8\textwidth]{binary_search_11}}\\\vspace{-8.22mm}
\uncover<13->{\includegraphics[width=0.8\textwidth]{binary_search_12}}\\\vspace{-8.22mm}
\uncover<14->{\includegraphics[width=0.8\textwidth]{binary_search_13}}\\\vspace{-8.22mm}
\uncover<15->{\includegraphics[width=0.8\textwidth]{binary_search_14}}\\\vspace{-8.22mm}
\end{center}

\end{frame}

%-----------------

\begin{frame}[fragile]{7.3.2~二分查找算法}

二分查找算法的实现如下:

\vspace{-4mm}

\begin{columns}[t]

\column{0.65\textwidth}
\begin{blueblock}{\texttt{Array}成员函数\texttt{binarySearch}定义}
\begin{lstlisting}[moreemph={Array,T,F}]
template<typename T, size_t N>
int Array<T, N>::binarySearch(const T &value, int left, int right) {
    while (left <= right) {
        int middle = (left + right) / 2;//计算中点位置
        if (m_ele[middle] == value)
            return middle;
        else if (m_ele[middle] > value)
            right = middle - 1;//修改right
        else
            left = middle + 1;//修改left
    }
    return -1; //查找失败
}
\end{lstlisting}
\end{blueblock}


\column{0.3\textwidth}
\begin{yellowblock}{说明}
$\bullet$ 如果\texttt{value}小于中点位置(\texttt{middle})元素,则将\texttt{right}设为\texttt{middle-1}\\
$\bullet$ 如果\texttt{value}大于中点位置元素,则将\texttt{left}设为\texttt{middle+1}\\
$\bullet$ 如果查找失败则返回\texttt{-1}
\end{yellowblock}
\vspace{-2mm}
\begin{greenblock}<2->{问题}
查找\texttt{4}返回时,\texttt{left}和\texttt{right}的值是多少?
\end{greenblock}
\vspace{-2mm}
\begin{greenblock}<3->{答案}
\texttt{left}为\texttt{2},\texttt{right}为\texttt{1}
\end{greenblock}

\end{columns}

\end{frame}

%-----------------

\begin{frame}[c]{~}
\begin{center}
  \huge{本章结束}
\end{center}
\end{frame}

%----------------------------------------------------------------------------------------

\end{document}
