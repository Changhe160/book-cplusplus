%!TEX program = xelatex
\documentclass[aspectratio=43,UTF8,10pt]{ctexbeamer}

    \mode<presentation> {
    \usetheme{Madrid}
    %\setbeamertemplate{footline} % To remove the footer line in all slides uncomment this line
    \setbeamertemplate{footline}[frame number] % To replace the footer line in all slides with a simple slide count uncomment this line
    \setbeamercolor{page number in head/foot}{fg=blue}
    \setbeamertemplate{navigation symbols}{} % To remove the navigation symbols from the bottom of all slides uncomment this line
    }
    \usepackage{indentfirst}
    \setlength{\parindent}{2em}
    \usepackage{listings}
    \lstset{language=C++, showstringspaces=false, basicstyle=\small}
    \usepackage{wrapfig}
    \usepackage{graphicx}
    % \usepackage{fboxrule}
\definecolor{hanblue}{rgb}{0.27, 0.42, 0.81}
\definecolor{indiagreen}{rgb}{0.07, 0.53, 0.03}
\definecolor{indianred}{rgb}{0.8, 0.36, 0.36}
\definecolor{indianyellow}{rgb}{0.89, 0.66, 0.34}
\definecolor{babypink}{rgb}{0.96, 0.76, 0.76}
\definecolor{ao(english)}{rgb}{0.0, 0.5, 0.0}

    \definecolor{azure(colorwheel)}{rgb}{0.0, 0.5, 1.0}
    \definecolor{alizarin}{rgb}{0.82, 0.1, 0.26}
    % User Defined Block %%%%%%%%%%%%%%%%%%%%%%%%%%%%%%%%%%%%%%%%%%%%%%%%%%%%%%%%




    \newenvironment<>{blueblock}[1]{%
      \setbeamercolor{block title}{fg=white,bg=hanblue}%
    %   \setbeamercolor{block body}{fg=white,bg=bondiblue}%
      \begin{block}#2{#1}}{\end{block}}

    \newenvironment<>{greenblock}[1]{%
    % \setbeamercolor{block title}{fg=white,bg=bondiblue}%
      \setbeamercolor{block title}{fg=white,bg=indiagreen}%
    \begin{block}#2{#1}}{\end{block}}

    \newenvironment<>{redblock}[1]{%
      \setbeamercolor{block title}{fg=white,bg=indianred}%
    %   \setbeamercolor{block body}{fg=white,bg=bondiblue}%
      \begin{block}#2{#1}}{\end{block}}

    \newenvironment<>{yellowblock}[1]{%
      \setbeamercolor{block title}{fg=white,bg=indianyellow}%
      \begin{block}#2{#1}}{\end{block}}

    \lstset{language=C++,
    columns=flexible,
    basicstyle=\footnotesize\ttfamily,                                    % 设定代码字体、大小
    %numbers=left,xleftmargin=2em,framexleftmargin=2em,                   % 在左侧显示行号
    %numberstyle=\color{darkgray},                                        % 设定行号格式
    keywordstyle=\color{blue},                                            % 设定关键字格式
    commentstyle=\color{ao(english)},                                     % 设置代码注释的格式
    stringstyle=\color{brown},                                            % 设置字符串格式
    showstringspaces=false,                                              % 控制是否显示空格
    %frame=single,                                                         % 控制外框
    breaklines,                                                           % 控制是否折行
    % postbreak=\space,                                                     % 控制折行后显示的标识字符
    breakindent=5pt,                                                      % 控制折行后缩进数量
    emph={size\_t,array,deque,list,map,queue,set,stack,vector,string,pair,tuple,ostream,istream}, % 非内置类型
    emphstyle={\color{teal}},
    escapeinside={(*@}{@*)},
}

\usepackage{multicol}

%----------------------------------------------------------------------------------------
%	TITLE PAGE
%----------------------------------------------------------------------------------------



\title[\textit{C++程序设计:总结上}]{总结与复习:第1-6章} % The short title appears at the bottom of every slide, the full title is only on the title page

%\author[李长河]{李长河} % Your name
%\institute[CUG] % Your institution as it will appear on the bottom of every slide, may be shorthand to save space
%{
%中国地质大学(武汉)\\ % Your institution for the title page
%\medskip
%\textit{lichanghe@cug.edu.cn} % Your email address
%}
\date{} % Date, can be changed to a custom date

\begin{document}

%----------------------------------------------------------------------------------------
%	TIKZ FLOWCHART
%----------------------------------------------------------------------------------------
%\tikzstyle{startstop} = [rectangle, rounded corners, minimum width=2cm, minimum height=0.5cm, text centered, draw=black, fill=red!30, font=\tiny]
%\tikzstyle{io} = [trapezium, trapezium left angle=70, trapezium right angle=110, minimum width=0cm, minimum height=0cm, text centered, draw=black, fill=blue!30, font=\tiny]
%\tikzstyle{process} = [rectangle, minimum width=2.5cm, minimum height=1.5cm, text centered, draw=black, fill=orange!30, font=\tiny, text width=2cm]
%\tikzstyle{decision} = [diamond, minimum width=2.5cm, minimum height=2cm, text centered, draw=black, fill=green!30, font=\tiny, text width=1.8cm, aspect=1.1]

\begin{frame}
\titlepage % Print the title page as the first slide

\begin{multicols}{2}
	黑色部分:基础知识,掌握		
	\color{blue}{蓝色部分:高级话题,了解}
\end{multicols}
\end{frame}

\begin{frame}[fragile]
	\frametitle{第一章:认识C++程序}
\begin{multicols}{2}
\begin{itemize}
  \item 程序基本组成
  \item main函数:有且仅有1个main函数
\item 注释语句:单行注释与多行注释
\item 输入输出:std::cin, std::cout
\item 编译与调试
\end{itemize}
\end{multicols}
\end{frame}



\begin{frame}[fragile]
	\frametitle{第二章:基本数据类型和表达式}
\begin{multicols}{2}
\begin{itemize}
  \item 基本元素:用户自定义标示符和关键字
\item 基本数据类型
\begin{itemize}
  \item 内置类型:含义和尺寸(字节)
  \item 字面值常量(右值):整型常量、实型常量和字符常量(转义字符)的表示方法(前缀和后缀)
\end{itemize}

\item 对象:数据和操作的载体
\begin{itemize}
  \item 定义和初始化,声明
  \item 类型、名字、内存结构、生命期、作用域
\end{itemize}

\item const修饰符和类型推导
\begin{itemize}
  \item const对象和\textcolor{blue}{constexpr}
  \item 类型别名:using或typedef
  \item {\color{blue}类型推导:auto和decltype}
\end{itemize}

\item 表达式求值与构造
\begin{itemize}
  \item {\color{blue}左值、右值}、优先级、结合性和{\color{blue}求值次序}
  \item 优先级:逗号运算符\texttt{<}赋值运算符~\texttt{<}逻辑或\texttt{<}逻辑与~\texttt{<}~关系运算符~\texttt{<}~算术运算符~\texttt{<}~逻辑非
      \item 条件运算符、sizeof运算符、\textcolor{blue}{位运算符}
  \item 表达式构造,短路求值,隐式类型转换规则
\end{itemize}
\end{itemize}
\end{multicols}
\end{frame}

\begin{frame}[fragile]
	\frametitle{第三章:语句控制结构}
\begin{multicols}{2}
\begin{itemize}
  \item 语句:分号结束的表达式语句
\begin{itemize}
  \item 空语句和语句块
  \item 语句作用域
\end{itemize}

\item 分支结构
\begin{itemize}
  \item if语句, if else语句,if嵌套,悬垂else
  \item switch语句:标签,break
  \item 循环语句:while, do while和for(执行流程)
  \item 循环语句选择
  \item 跳转语句:break只能用于循环或switch语句;continue只能用于循环结构
\end{itemize}

\item 嵌套结构:根据问题分析设计
\end{itemize}
\end{multicols}
\end{frame}

\begin{frame}[fragile]
	\frametitle{第四章:复合类型、string和vector}

\begin{multicols}{2}
\begin{itemize}
  \item (左值)引用:左值对象的别名
\begin{itemize}
  \item 定义时必须初始化
  \item 指向const对象的引用
  \item {\color{blue}auto, decltype和引用}
  \item {\color{blue}右值引用:把右值变成左值}
\end{itemize}

\item 指针:通过对象的地址访问对象内容
\begin{itemize}
  \item 指针对象:存储左值对象的地址
  \item 赋值操作改变指针指向
  \item 指向const对象的指针域const指针
  \item {\color{blue}auto, decltype和指针}
  \item {\color{blue}void指针和多级指针}
  \item 指针和引用:初始化、赋值行为不同;引用可以看做const指针
\end{itemize}

\item 数组:有限个同类型元素的有序集合
\begin{itemize}
  \item 初始化(字符数组)
  \item {\color{blue}复杂数组定义}
  \item 下标法访问;{\color{blue}范围for语句}
  \item 二维数组
\end{itemize}

\item 指针和数组:利用指针访问数组(地址、元素和值)
\begin{itemize}
  \item 指针指向数组
  \item 指向数组的指针运算:移动、关系和减法
  \item 访问二维数组:*(*(p2d+1)+1),*(p2d[1]+1)
\end{itemize}

\item {\color{blue}string、vector和枚举类型}
\item {\color{blue}C风格字符数组的处理函数}
\end{itemize}
\end{multicols}
\end{frame}


\begin{frame}[fragile]
	\frametitle{第五章:函数}
\begin{multicols}{2}
\begin{itemize}
  \item 函数,代码间数据传递和交互的主要方式
\begin{itemize}
  \item 函数四个要素
  \item 函数声明与函数调用
\end{itemize}

\item 存储类型
\begin{itemize}
  \item 局部对象:栈,无链接性
  \item 全局对象:全局数据区,外部链接性
  \item 静态类型:无外部链接性
\end{itemize}

\item 参数传递
\begin{itemize}
  \item 单向值传递,安全但低效 Swap(int,int);
  \item 地址传递:传递实参的地址 Swap(int *, int*);
  \item 引用传递:形参与实参指向同一个存储空间 Swap(int\&,int\&);
  \item 兼顾效率与安全: const 引用
  \item 数组形参:地址方式传递,需要传递数组长度(字符数组例外)
\end{itemize}

\item 返回值类型
\begin{itemize}
  \item 值返回:右值临时对象
  \item 引用或地址返回:左值(在函数调用之前产生),不要返回函数内部在栈里产生的局部对象的引用或地址
  \item \color{blue}{函数重载、形参默认值、内联函数、函数指针、lambda表达式}
  \item \color{blue}{递归程序设计}
\end{itemize}

\item 编译预处理和多文件结构
\begin{itemize}
  \item 宏定义:带参和不带参
  \item \color{blue}{条件编译}
  \item \color{blue}{多文件结构:头文件和源文件}
\end{itemize}

\end{itemize}
\end{multicols}
\end{frame}


\begin{frame}[fragile]
	\frametitle{第六章:类}
\begin{multicols}{2}
\begin{itemize}
  \item 类的定义
  \begin{itemize}
    \item 数据成员和成员函数的封装\\ class Fraction\{ \\
    private: 数据成员;\\
    public: 成员函数; \\
    \};
    \item 成员函数调用 :. ->
    \item 访问控制和友元
  \end{itemize}
  \item 构造与析构
  \begin{itemize}
    \item 构造函数:语法和功能,初始化列表
    \item 默认与复制构造函数:直接与复制初始化
    \item 析构函数:语法和功能
  \end{itemize}
\end{itemize}

\begin{itemize}
  \item 运算符重载
  \begin{itemize}
    \item 重载原则:含义与行为与内置类型一致;改变自身状态的作为类成员,对称性的作为辅助函数
    \item 输入和输出运算符:辅助函数,注意返回值和参数类型
    \item \color{blue}{递增和递减运算符:成员函数,注意行为与内置类型一致}
    \item \color{blue}{函数调用和类型转换运算符}
  \end{itemize}
 \item 静态成员
\begin{itemize}
  \item 静态数据成员类内声明类外初始化
  \item 静态成员函数不能访问非静态数据成员
  \item 类名::成员名
\end{itemize}
\end{itemize}
\end{multicols}
\end{frame}


\begin{frame}[fragile]
	\frametitle{算法}
\begin{multicols}{2}
\begin{itemize}
  \item 直接法:例3.2(成绩转换)、3.3(解方程)、3.5(字符统计)、3.6(猜数字)、3.8(石头剪刀布)、3.9(百钱买百鸡)、4.1(分数统计)、4.3(扫雷地图)、4.4(猜单词)
  \item 迭代法:例3.4(计算$\pi$)、3.7(打印sin函数)
  \item 回溯法(深度优先):例4.2(八皇后)
  \item 宽度优先:例4.5(扫雷连通域)
  \item \textcolor{blue}{递归方法:例5.5(汉诺塔)、5.6(八皇后)}
  \item 其它应用:例5.2(排序),5.3(求积分)
\end{itemize}
\end{multicols}
\end{frame}


\end{document}
